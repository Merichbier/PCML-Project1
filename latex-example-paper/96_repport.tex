\documentclass[10pt,conference,compsocconf]{IEEEtran}

\usepackage{hyperref}
\usepackage{graphicx}	% For figure environment


\begin{document}
\title{Machine Learning Course - Project 1}

\author{
  Victor Faramond, Mathieu Schopfer, Dario Anongba Varela\\
  \textit{Department of Computer Science, EPFL Lausanne, Switzerland}
}

\maketitle

\begin{abstract}
\end{abstract}

\section{Introduction}

\section{Models and Methods}

\subsection{Data}

\subsubsection{Available data}
Two sets of data were provided in the form of csv files, a \textit{training} and a \textit{test} set. The input of both sets consists in $D=30$ feature columns $\mathbf{x}_n$. In the training set, an additionnal output column $y_n$ is provided. The output is binary (-1 or 1), depending whether the recorded signal corresponds to an actual event (1) or to background noise (-1).

\subsubsection{Data pre-processing}
Impute missing data: for a given input column $i \in {1, ..., D}$, missing values are replaced in both the training and test sets by the value appearing most frequently in column $i$ of the training set.

Standardize

Compute inverse log of positive features

\subsection{Models}
\subsubsection{Gradient and stochastic gradient descent}
\subsubsection{Least squares}
\subsubsection{Ridge regression}
Polynom building
\subsubsection{Logistic and penalized logistic regression}

\subsection{Cross validation}

To train and test the various models presented hereabove, a 10-fold cross validation method was used. This consists in splitting the training data set into ten subsets of equivalent size. Then 9 subsets are used to train the models, the last one to test it. This proces is repeated for all 10 combinations of training and testing subsets.

\subsection{Assement of models accuracy}

At each step of the 10-fold cross validation, the positives rate was calculated on the test subset. The positive rates is simply given as the number of model ouptuts $y_n$ agreeing with the (known) expected output divided by the total number of events in the subset.

Then the mean and variance of positives rates was calculated and retained as the model accuracy score.

\subsection{Choice of model for submission}

The model that was choosen for submission was that achieving the highest positives rate score.

\section{Results}



\section{Summary}


\end{document}
